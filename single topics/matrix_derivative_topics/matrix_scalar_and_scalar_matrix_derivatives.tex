    \subsection*{X.14 \quad Derivative: matrix/scalar}
    
        Now, we have our general form for creating derivatives.
        
        We'll get our derivative of the form 
        
        \begin{equation}
            \pderiv{ \text{(Matrix)} } { \text{(Scalar) } }
            =
            \pderiv{ \blu{M} }{ \red{s} } 
        \end{equation}
        
        We have a matrix $\blu{M}$ in the shape $(r \times k)$ and a scalar $\red{s}$. Our \textbf{input} is a \textbf{scalar}, and our \textbf{output} is a \textbf{matrix}.
        
        \begin{equation}
            \blu{M}
            =
            \begin{bmatrix}
                \blu{m}_{11} & \blu{m}_{12} & \cdots & \blu{m}_{1r} \\ 
                \blu{m}_{21} & \blu{m}_{22} & \cdots & \blu{m}_{2r} \\ 
                \vdots       & \vdots       & \ddots & \vdots     \\
                \blu{m}_{k1} & \blu{m}_{k2} & \cdots & \blu{m}_{kr} \\ 
            \end{bmatrix}
        \end{equation}
        
        This may seem concerning: before, we divided \textbf{inputs} across \textbf{rows}, and \textbf{outputs} across \textbf{columns}. But in this case, we have \textbf{no} input axes, and \textbf{two} output axes. 
        
        Well, let's try to make this work anyway. 
        
        What did we do before, when we didn't know how to handle a \textbf{new} derivative? We compared it to \textbf{old} versions: we built our vector/vector case using the vector/scalar case and the scalar/vector case.
        
        We did this by \textbf{compressing} one of our \textit{vectors} into a \textit{scalar} temporarily: this works, because we want to treat each of these objects the \textbf{same way}.
        
        We don't know how to work with Matrix/Scalar, but what's the \textbf{closest} thing we do know? \textbf{Vector/Scalar}.
        
        How do we accomplish that? As we saw above, a matrix is a \textbf{vector} of \textbf{vectors}. We could turn it into a \textbf{vector} of \textbf{scalars}.\\
        
        \begin{concept}
            A \vocab{matrix} can be thought of as a \orgg{column vector} of \purp{row vectors} (or vice versa).
            
            So, we can use our earlier technique and convert the \purp{row vectors} into \redd{scalars}.
        \end{concept}
        
        We'll replace the \textbf{row vectors} in our matrix with \textbf{scalars}. 
        
        \begin{equation}
            \blu{M}
            =
            \begin{bmatrix}
                \blu{M}_1 \\ \blu{M}_2 \\ \vdots \\ \blu{M}_k
            \end{bmatrix}
        \end{equation}
    
        Now, we can pretend our matrix is a vector! We've got a derivative for that:
        
        \begin{equation}
            \pderiv{ \blu{M} }{ \red{s} } 
            =
            \begin{bmatrix}
                \bigpderiv{ \blu{M_1} }{ \red{s} } &
                \bigpderiv{ \blu{M_2} }{ \red{s} } &
                \cdots &
                \bigpderiv{ \blu{M_r} }{ \red{s} } 
            \end{bmatrix}
        \end{equation}
        
        Aha - we have the same form that we did for our vector/vector derivative! Each derivative is a column vector. Let's expand it out:
        
        \begin{equation}
            \pderiv{ \blu{M} }{ \red{s} } 
            =
            \overbrace{
                \begin{bmatrix}
                    \begin{bmatrix}
                        \bigpderiv{ \blu{m}_{11} }   { \red{s} }\\ 
                        \\
                        \bigpderiv{ \blu{m}_{12} }   { \red{s} }\\ 
                        \\
                        \vdots \\ 
                        \\
                        \bigpderiv{ \blu{m}_{1r} }   { \red{s} }
                    \end{bmatrix}, &
                    \begin{bmatrix}
                        \bigpderiv{ \blu{m}_{21} }   { \red{s} }\\ 
                        \\
                        \bigpderiv{ \blu{m}_{22} }   { \red{s} }\\ 
                        \\
                        \vdots \\ 
                        \\
                        \bigpderiv{ \blu{m}_{2r} }   { \red{s} }
                    \end{bmatrix}, &
                    \cdots &
                    \begin{bmatrix}
                        \bigpderiv{ \blu{m}_{k1} }   { \red{s} }\\ 
                        \\
                        \bigpderiv{ \blu{m}_{k2} }   { \red{s} }\\ 
                        \\
                        \vdots \\ 
                        \\
                        \bigpderiv{ \blu{m}_{kr} }   { \red{s} }
                    \end{bmatrix}
                \end{bmatrix}
            }^{ \text{\small Column $j$ matches $\blu{m}_{j?}$} }
            \bigggrB{125pt} \text{\small Row $i$ matches $\blu{m}_{?i}$} 
        \end{equation}
        
        \begin{definition}
            If $\blu{M}$ is a matrix in the shape $\blu{ (r \times k) }$ and $\red{s}$ is a scalar,
            
            Then we define the \vocab{matrix derivative} $\pderivslash{ \blu{M} }{ \red{s} }$ as the $\blu{ (k \times r) }$ matrix:
            
            \begin{equation*}
                \pderiv{ \blu{M} }{ \red{s} } 
                =
                \overbrace{
                    \begin{bmatrix}
                        \begin{matrix}
                            \bigpderiv{ \blu{m}_{11} }   { \red{s} }\\ 
                            \\
                            \bigpderiv{ \blu{m}_{12} }   { \red{s} }\\ 
                            \\
                            \vdots \\ 
                            \\
                            \bigpderiv{ \blu{m}_{1r} }   { \red{s} }
                        \end{matrix} 
                        &
                        \begin{matrix}
                            \bigpderiv{ \blu{m}_{21} }   { \red{s} }\\ 
                            \\
                            \bigpderiv{ \blu{m}_{22} }   { \red{s} }\\ 
                            \\
                            \vdots \\ 
                            \\
                            \bigpderiv{ \blu{m}_{2r} }   { \red{s} }
                        \end{matrix}
                        &
                        \begin{matrix}
                            \cdots\\\\ \cdots \\\\ \ddots \\\\ \cdots
                        \end{matrix} 
                        &
                        \begin{matrix}
                            \bigpderiv{ \blu{m}_{k1} }   { \red{s} }\\ 
                            \\
                            \bigpderiv{ \blu{m}_{k2} }   { \red{s} }\\ 
                            \\
                            \vdots \\ 
                            \\
                            \bigpderiv{ \blu{m}_{kr} }   { \red{s} }
                        \end{matrix}
                    \end{bmatrix}
                }^{ \text{\small Column $j$ matches $\blu{m}_{j?}$} }
                \bigggrB{125pt} \text{\small Row $i$ matches $\blu{m}_{?i}$} 
            \end{equation*}
            
            This matrix has the transpose of the shape of $\blu{M}$.
        \end{definition}
        
    \subsection*{X.15 \quad Derivative: scalar/matrix}
    
        We'll get our derivative of the form 
        
        \begin{equation}
            \pderiv{ \text{(Scalar)} } { \text{(Matrix)} } 
            =
            \pderiv{ \red{s} }{ \blu{M} } 
        \end{equation}
        
        We have a matrix $\blu{M}$ in the shape $(r \times k)$ and a scalar $\red{s}$. Our \textbf{input} is a \textbf{matrix}, and our \textbf{output} is a \textbf{scalar}.
        
        Let's do what we did last time: break it into \textbf{row vectors}.
        
        \begin{equation}
            \blu{M}
            =
            \begin{bmatrix}
                \blu{M}_1 \\ \blu{M}_2 \\ \vdots \\ \blu{M}_k
            \end{bmatrix}
        \end{equation}
        
        The gradient for this "vector" gives us a \textbf{column vector}:
        
        \begin{equation}
            \pderiv{ \red{s} }{ \blu{M} }  
            =
            \begin{bmatrix}
                \bigpderiv{ \red{s} }{ \blu{M_1} } \\
                \\
                \bigpderiv{ \red{s} }{ \blu{M_2} } \\
                \\
                \vdots \\
                \\
                \bigpderiv{ \red{s} }{ \blu{M_k} }
            \end{bmatrix}
        \end{equation}
        
        This time, each derivative is a \textbf{row vector}. Let's \textbf{expand}:
        
        \begin{equation}
            \pderiv{ \red{s} }{ \blu{M} }  
            =
            \begin{bmatrix}
                \begin{bmatrix}
                    \bigpderiv{ \red{s} }{ \blu{m}_{11} }    
                    & 
                    \bigpderiv{ \red{s} }{ \blu{m}_{12} } 
                    & 
                    \cdots 
                    & 
                    \bigpderiv{ \red{s} }{ \blu{m}_{1r} } 
                \end{bmatrix}
                \\\\
                \begin{bmatrix}
                    \bigpderiv{ \red{s} }{ \blu{m}_{21} }    
                    & 
                    \bigpderiv{ \red{s} }{ \blu{m}_{22} } 
                    & 
                    \cdots 
                    & 
                    \bigpderiv{ \red{s} }{ \blu{m}_{2r} } 
                \end{bmatrix}
                \\\\
                \vdots
                \\\\
                \begin{bmatrix}
                    \bigpderiv{ \red{s} }{ \blu{m}_{k1} }    
                    & 
                    \bigpderiv{ \red{s} }{ \blu{m}_{k2} } 
                    & 
                    \cdots 
                    & 
                    \bigpderiv{ \red{s} }{ \blu{m}_{kr} } 
                \end{bmatrix}
            \end{bmatrix}
        \end{equation}
        
        
        \begin{definition}
            If $\blu{M}$ is a matrix in the shape $\blu{ (r \times k) }$ and $\red{s}$ is a scalar,
            
            Then we define the \vocab{matrix derivative} $\pderivslash{ \red{s} }{ \blu{M} }$ as the $\blu{ (r \times k) }$ matrix:
            
            \begin{equation*}
                \pderiv{ \red{s} }{ \blu{M} }
                =
                \overbrace{
                    \begin{bmatrix}
                        \begin{matrix}
                            \bigpderiv{ \red{s} } { \blu{m}_{11} }\\ 
                            \\
                            \bigpderiv{ \red{s} } { \blu{m}_{21} }\\ 
                            \\
                            \vdots \\ 
                            \\
                            \bigpderiv{ \red{s} } { \blu{m}_{k1} }
                        \end{matrix} 
                        &
                        \begin{matrix}
                            \bigpderiv{ \red{s} } { \blu{m}_{12} }\\ 
                            \\
                            \bigpderiv{ \red{s} } { \blu{m}_{22} }\\ 
                            \\
                            \vdots \\ 
                            \\
                            \bigpderiv{ \red{s} } { \blu{m}_{k2} }
                        \end{matrix}
                        &
                        \begin{matrix}
                            \cdots\\\\ \cdots \\\\ \ddots \\\\ \cdots
                        \end{matrix} 
                        &
                        \begin{matrix}
                            \bigpderiv{ \red{s} } { \blu{m}_{1r} }\\ 
                            \\
                            \bigpderiv{ \red{s} } { \blu{m}_{2r} }\\ 
                            \\
                            \vdots \\ 
                            \\
                            \bigpderiv{ \red{s} } { \blu{m}_{kr} }
                        \end{matrix}
                    \end{bmatrix}
                }^{ \text{\small Column $j$ matches $\blu{m}_{?j}$} }
                \bigggrB{125pt} \text{\small Row $i$ matches $\blu{m}_{i?}$} 
            \end{equation*}
            
            This matrix has the same shape as $\blu{M}$.
        \end{definition}
        
        