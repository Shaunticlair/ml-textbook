\section*{Multiple Data Points in a Matrix}




    \subsection*{Summing over data points}
    
        Currently, when using our \textbf{objective} function, we have to \textbf{sum} over \textbf{every} single data point. For the 1D case, this means we have to do:
        
        \begin{equation}
            J = 
            \frac{1}{n}  \sum_{i=1}^n \red{(\theta \ex{x}{i}  - \ex{y}{i} )^2} 
        \end{equation}
        
        This is a bit of a hassle - it \textbf{forces} us to use $\ex{x}{i}$ notation, and we have to be conscious of that \textbf{sum}.
        
        By using \textbf{vectors} above, we were able to work with \textbf{many} variables $\theta_k$ at the same time, making it easier to \textbf{represent} and \textbf{work} with them in the future. 
        
        Can we do the \textbf{same} here - combining many \textbf{data points} into one object, rather than many \textbf{variables}?
    
    \subsection*{Summing with Vectors: Row Vectors}
        
        We want to represent \textbf{addition} using \textbf{vectors}. We did that when we were adding $x_k\theta_k$ terms with a \textbf{dot product}.
        
        \begin{equation}
            \red{\theta_0} + \red{\theta_1}x_1 + \red{\theta_2}x_2 + \red{\theta_3}x_3 + ...
            =
            \begin{bmatrix}
              \red{\theta_0} \\ \theta_1 \\ \theta_2 \\ \theta_3 \\ \vdots \\ \theta_d
            \end{bmatrix}
            \cdot
            \begin{bmatrix}
              \red{1} \\ x_1 \\ x_2 \\ x_3 \\ \vdots \\ x_d
            \end{bmatrix}
        \end{equation}
        
        But, dot products also include \textbf{multiplication}. Above, our terms are \textbf{squared}. So, we can multiply $(\theta \ex{x}{i} - \ex{y}{i} )$ times itself!
        
        \begin{equation}
            J = 
            \frac{1}{n}  \sum_{i=1}^n 
            (\theta \ex{x}{i}  - \ex{y}{i} ) (\theta \ex{x}{i}  - \ex{y}{i} )  
        \end{equation}
        
        We'll write $\ex{r}{i} = \theta \ex{x}{i}  - \ex{y}{i}$ to simplify our work. 
        \begin{equation}
            J = \frac{1}{n}  \sum_{i=1}^n
            \ex{r}{i}*\ex{r}{i}
        \end{equation}
        
        In a dot product, we \textbf{add} the \textbf{dimensions} together. So, we'll give each term in our sum its own \textbf{dimension}.
        
        \begin{equation}
            J = \frac{1}{n}  \sum_{i=1}^n
            \ex{r}{i}*\ex{r}{i}
            =
            \frac{1}{n}
            \begin{bmatrix}
              \ex{r}{1} \\ \ex{r}{2} \\ \ex{r}{3} \\ \vdots \\ \ex{r}{d}
            \end{bmatrix}
            \cdot
            \begin{bmatrix}
              \ex{r}{1} \\ \ex{r}{2} \\ \ex{r}{3} \\ \vdots \\ \ex{r}{d}
            \end{bmatrix}
        \end{equation}
        
        We've got a single vector we could call $R$.
        
        We could make it a \textbf{column vector}, but we already use the \textbf{rows} to indicate the \textbf{dimensions}. 
        
        \begin{equation}
            \begin{drcases}
                \theta =
                \begin{bmatrix}
                  \red{\theta_0} \\ \theta_1 \\ \theta_2 \\ \theta_3 \\ \vdots \\ \theta_d
                \end{bmatrix}
            \end{drcases}
            \text{dimensions as rows...}
        \end{equation}
        
        So, let's use \textbf{columns} instead: each \textbf{column} will be a \textbf{data point}: we'll use a \textbf{row vector} $(1 \times n)$. 
        
        \begin{equation}
            R =
            \overbrace{
            \begin{bmatrix}
              \ex{r}{1} & \ex{r}{2} & \ex{r}{3} & \cdots & \ex{r}{d} 
            \end{bmatrix}
            }^{\text{data points as columns!}}
        \end{equation}
        
    \subsection{Going from $x$ to $X$}
    
        We can do the same for our input data $\ex{x}{i}$:\\
        
        \begin{notation}
            We can store all of our 1-D \vocab{data points} in a \vocab{row vector}:
            
            \begin{equation*}
                X =
                \begin{bmatrix}
                  \ex{x}{1} & \ex{x}{2} & \ex{x}{3} & \cdots & \ex{x}{d}
                \end{bmatrix}
            \end{equation*}
            \begin{equation*}
                Y =
                \begin{bmatrix}
                  \ex{y}{1} & \ex{y}{2} & \ex{y}{3} & \cdots & \ex{y}{d}
                \end{bmatrix}
            \end{equation*}
        \end{notation}
        
        Since we had $\ex{r}{i} = (\theta \ex{x}{i} - \ex{y}{i} )$, we can write 
        
        \begin{equation}
            R = \theta X - Y
        \end{equation}
        \note{Still in the 1D case!}
        
        We can write our \textbf{objective function} as
        
        \begin{equation}
            J = 
            \frac{1}{n}
            \begin{bmatrix}
              \ex{r}{1} & \ex{r}{2} & \ex{r}{3} & \cdots & \ex{r}{d}
            \end{bmatrix}
            \begin{bmatrix}
              \ex{r}{1} \\ \ex{r}{2} \\ \ex{r}{3} \\ \vdots \\ \ex{r}{d}
            \end{bmatrix}
        \end{equation}
        
        So, we can write compactly:
        
        \begin{equation}
            J = \frac{1}{n} RR^T
        \end{equation}
        
        Let's expand this back out with $R = \theta X - Y$:\\
        
        \begin{concept}
            In 1-D, we can use row vectors to sum our data points as
            
            \begin{equation*}
                J = \frac{1}{n} (\theta X - Y)(\theta X - Y)^T
            \end{equation*}
        \end{concept}
        
        We've successfully \textbf{removed} the \textbf{sum}!
        
        This format \textbf{stores} all of our \textbf{data points} in \textbf{one object}, just like how we wanted.
        