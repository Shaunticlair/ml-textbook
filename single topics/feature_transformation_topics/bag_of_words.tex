        \subsection*{Text: Bag of Words}


            Another very common data type we work with is \textbf{language}: bodies of text, online articles, corpora, etc.

            Later in this course, we will discuss more powerful ways to analyze text, such as \textbf{sequential models}, and \textbf{transformers}.
                \note{Obligatory chatgpt reference.}

            There's a very simple encoding that we'll focus on here: the \textbf{bag of words} approach.

            This approach is meant to be as simple as possible: for each word, we ask ourselves, "if this word in the text?", and answer yes (1) or no (0) for every single word.\\

            \begin{definition}
                The \vocab{bag of words} feature transformation takes a body of text, and creates a \purp{feature} for every \gren{word}: is that word in the text, or not?

                \begin{equation}
                    \phi(x) = 
                    \begin{bmatrix}
                        \text{Word 1 in } x \\
                        \text{Word 2 in } x \\
                        \vdots \\
                        \text{Word k in } x \\
                    \end{bmatrix}
                \end{equation}

                This approach is used for \purp{bodies of text}.
            \end{definition}

            \miniex Consider the following sentence: "She read a book."

            With the words: $\{She, he, a, read, tired, water, book\}$

            We get:

            \begin{equation}
                \phi(\text{"She read a book."}) =
                \begin{bmatrix}
                    1& 0& 1& 1& 0& 0& 1
                \end{bmatrix}
            \end{equation}

            A couple weaknesses to this approach:

            \begin{itemize}
                \item Ignores the order of words and syntax of the sentence.
                \item Doesn't encode meaning directly.
                \item Duplicate words are only included once.
                \item It doesn't create much structure for our model to use.
            \end{itemize}

            But, it's very easy to implement.
