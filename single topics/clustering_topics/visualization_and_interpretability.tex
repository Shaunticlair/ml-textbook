    \subsection*{Applications: Visualization and Interpretability}
    
        We've discussed some ways to \textbf{abstractly} test whether our clustering might be \textbf{accurate} the data.
        
        But, when it comes down to it, often, the \textbf{quality} of a clustering is based on how \textbf{useful} it is. So, what sorts of \textbf{uses} does clustering \textbf{have}? 
        
        Well, we're organizing our data into \textbf{groups}: this \textbf{simplifies} how we look at our data. And it allows us to \textbf{view} at our data, and \textbf{understand} what's going on.
        
        In short: clustering allows humans to more easily make sense of data.\\
        
        \begin{concept}
            One of the the main \gren{goals} of \vocab{clustering} is to make it easier for humans to \purp{understand} the data.
            
            This happens in two ways:
            
            \begin{itemize}
                \item We can \vocab{visualize} the data: we can \gren{see} it, and more easily use our \purp{intuition} to make sense of it.
                
                \item We can \vocab{interpret} our data: by seeing what sorts of \gren{groupings} we create, we learn about the \purp{structure} of the data.
            \end{itemize}
        \end{concept}
        
        So, machine learning experts judge partly based on how well a clustering \textbf{helps} them \textbf{achieve} these two goals.
        
        \textbf{Evaluating} clusterings is \textbf{subjective} for exactly this reason: what is \textbf{good} "visually", or is the \textbf{best} "interpretation" of data, is often up to \textbf{debate}.
        
        So, \textbf{human} judgement is important for this type of \textbf{problem}.
    