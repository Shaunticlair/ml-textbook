
\usepackage{amsmath,amsthm,amssymb,graphicx,mathtools,tikz,hyperref,mathdots}

\usepackage[normalem]{ulem}

%Not sure of any special consequences - used to fix \mathbb{1}
\usepackage[bb=dsserif]{mathalpha}

\usetikzlibrary{positioning}
\newcommand{\n}{\mathbb{N}}
\newcommand{\z}{\mathbb{Z}}
\newcommand{\q}{\mathbb{Q}}
\newcommand{\cx}{\mathbb{C}}
\newcommand{\real}{\mathbb{R}}
\newcommand{\field}{\mathbb{F}}
\newcommand{\ita}[1]{\textit{#1}}
\newcommand{\com}[2]{#1\backslash#2}
\newcommand{\oneton}{\{1,2,3,...,n\}}
\newcommand\idea[1]{\begin{gather*}#1\end{gather*}}
\newcommand\ef{\ita{f} }
\newcommand\eff{\ita{f}}
\newcommand\proofs[1]{\begin{proof}#1\end{proof}}
\newcommand\inv[1]{#1^{-1}}
\newcommand\setb[1]{\{#1\}}
\newcommand\en{\ita{n }}
\newcommand{\vbrack}[1]{\langle #1\rangle}





\newenvironment{drytheorem}[2][dryTheorem]{\begin{trivlist}
\item[\hskip \labelsep {\bfseries #1}\hskip \labelsep {\bfseries #2.}]}{\end{trivlist}}
\newenvironment{drylemma}[2][dryLemma]{\begin{trivlist}
\item[\hskip \labelsep {\bfseries #1}\hskip \labelsep {\bfseries #2.}]}{\end{trivlist}}
\newenvironment{dryexercise}[2][dryExercise]{\begin{trivlist}
\item[\hskip \labelsep {\bfseries #1}\hskip \labelsep {\bfseries #2.}]}{\end{trivlist}}
\newenvironment{dryreflection}[2][dryReflection]{\begin{trivlist}
\item[\hskip \labelsep {\bfseries #1}\hskip \labelsep {\bfseries #2.}]}{\end{trivlist}}
\newenvironment{dryproposition}[2][dryProposition]{\begin{trivlist}
\item[\hskip \labelsep {\bfseries #1}\hskip \labelsep {\bfseries #2.}]}{\end{trivlist}}
\newenvironment{drycorollary}[2][dryCorollary]{\begin{trivlist}
\item[\hskip \labelsep {\bfseries #1}\hskip \labelsep {\bfseries #2.}]}{\end{trivlist}}
 \hypersetup{
 colorlinks,
 linkcolor=blue
 }
 
%Used for arbitrary-sized curly braces
\newcommand{\biggglB}[1]{\left\{\!\parbox{0pt}{\rule{0pt}{#1}}\right.}
\newcommand{\bigggrB}[1]{\left.\parbox{0pt}{\rule{0pt}{#1}}\!\right\}}
 
 
\newcommand*\circled[1]{\tikz[baseline=(char.base)]{
            \node[shape=circle,draw,inner sep=2pt] (char) {#1};}}
 
\newcommand\overmat[2]{%
  \makebox[0pt][l]{$\smash{\color{white}\overbrace{\phantom{%
    \begin{matrix}#2\end{matrix}}}^{\text{\color{black}#1}}}$}#2}
\newcommand\bovermat[2]{%
  \makebox[0pt][l]{$\smash{\overbrace{\phantom{%
    \begin{matrix}#2\end{matrix}}}^{\text{#1}}}$}#2}
 
 
 
% \newcommand\iddots{\mathinner{
%   \kern1mu\raise1pt{.}
%   \kern2mu\raise4pt{.}
%   \kern2mu\raise7pt{\Rule{0pt}{7pt}{0pt}.}
%   \kern1mu
% }} 
 

%%Shaunticlair's own macros
 
\newcommand{\deriv}[2]{\frac{\dd #1}{\dd #2}}
\newcommand{\dderiv}[2]{\frac{\dd^2 #1}{\dd #2^2} }

\newcommand{\derivslash}[2]{{\dd #1}/{\dd #2}}
\newcommand{\pderivslash}[2]{{\partial #1}/{\partial #2}}


\newcommand{\pderiv}[2]{\frac{\partial #1}{\partial #2}}
\newcommand{\pdderiv}[2]{\frac{\partial^2 #1}{\partial #2^2}}

\newcommand{\bigderiv}[2]{\dfrac{\dd #1}{\dd #2}}
\newcommand{\bigpderiv}[2]{\dfrac{\partial #1}{\partial #2}}

\usepackage{parskip, physics, enumitem}

\definecolor{betterpurple}{HTML}{7F00FF}
\definecolor{darkbrown}{HTML}{964b00}
\definecolor{easypink}{HTML}{ff607f}
\definecolor{darkgray}{HTML}{6b7283}
\definecolor{darkgreen}{HTML}{008600}
\definecolor{brownbrown}{HTML}{A4711E}
\definecolor{teal}{HTML}{008080}
\definecolor{vlightblue}{HTML}{00d2ff}

\newcommand{\org}[1]{\textcolor{orange}{#1}}
\newcommand{\pur}[1]{\textcolor{betterpurple}{#1}}
\newcommand{\red}[1]{\textcolor{red}{#1}}
\newcommand{\blu}[1]{\textcolor{blue}{#1}}
\newcommand{\bro}[1]{\textcolor{darkbrown}{#1}}
\newcommand{\pin}[1]{\textcolor{easypink}{#1}}
\newcommand{\gre}[1]{\textcolor{darkgray}{#1}}
\newcommand{\grn}[1]{\textcolor{green!70!black}{#1}}
\newcommand{\tea}[1]{\textcolor{teal}{#1}}
\newcommand{\bru}[1]{\textcolor{brownbrown}{#1}}
\newcommand{\lbl}[1]{\textcolor{vlightblue}{#1}}

\newcommand{\orgg}[1]{\textbf{\textcolor{orange}{#1}}}
\newcommand{\redd}[1]{\textbf{\textcolor{red}{#1}}}
\newcommand{\gren}[1]{\textbf{\textcolor{green!70!black}{#1}}}
\newcommand{\brow}[1]{\textbf{\textcolor{darkbrown}{#1}}}
\newcommand{\purp}[1]{\textbf{\textcolor{betterpurple}{#1}}}
\newcommand{\brun}[1]{\textbf{\textcolor{brownbrown}{#1}}}
\newcommand{\pink}[1]{\textbf{\textcolor{easypink}{#1}}}
\newcommand{\lblu}[1]{\textbf{\textcolor{vlightblue}{#1}}}



\newcommand{\miniex}[0]{\purp{Example: }}
\newcommand{\nth}[1]{ #1^{ \text{th} } }

\newcommand{\In}[0]{\textbf{in }}

\newcommand{\rgi}[0]{ \red{\ex{g}{i}} }

\newcommand{\exi}[0]{ \ex{x}{i} }
\newcommand{\eyi}[0]{ \ex{y}{i} }

\newcommand{\rxi}[0]{ \red{\exi }}
\newcommand{\ryi}[0]{ \red{\eyi }}

\newcommand{\bxi}[0]{ \blu{\exi }}
\newcommand{\byi}[0]{ \blu{\eyi }}

\newcommand{\cmul}[0]{\; \cdot \;}

\newcommand{\secdiv}[0]{\centerline{\uwave{\uwave{\hspace{14cm}}}}}
\newcommand{\subsecdiv}[0]{\centerline{\uwave{\uwave{\hspace{10cm}}}}}
%\newcommand{\boxdiv}[0]{\centerline{\rule{10cm}{1pt}}}

\newcommand{\boxdiv}[0]{\centerline{\uwave{\uwave{\hspace{10cm}}}}}

\newcommand{\setty}[1]{\Bigl\{#1\Bigr\}}
\newcommand{\prob}[1]{ \mathbf{P} \Bigl\{#1\Bigr\}}
\newcommand{\given}[2]{ \mathbf{P} \Bigl\{#1 \;\; \Big| \;\; #2 \Bigr\}}

\newcommand{\smashedoverbrace}[2]{\smash{\overbrace{#1}^{#2}}\vphantom{#1}}
\newcommand{\smashedunderbrace}[2]{\smash{\underbrace{#1}_{#2}}\vphantom{#1}}

% \newcommand{\loss}[0]{\mathcal{L}}

%%% Local Variables:
%%% mode: latex
%%% TeX-master: "top"
%%% End:
